%Fiquemos com Deus e Nossa Senhora!
%Sao Jose de Cupertino rogai por nos!!
% ### Uses XeLaTeX ### %
% ### Needs beamer-master ### %
\documentclass[aspectratio=169]{beamer} %. Aspect Ratio 16:9

\usetheme{AI2} % beamerthemeSprace.sty
\usepackage[portuguese]{babel}
\usepackage[utf8]{inputenc}
\usepackage[T1]{fontenc}
\usepackage{ragged2e,gensymb}

\DeclareMathOperator*{\argmin}{arg\,min}
\DeclareMathOperator*{\argmax}{arg\,max}
\DeclareMathOperator{\sign}{sgn}

% DATA FOR FOOTER
\date{2021}
\title{- Modelo de Mistura de Gaussianas}
\author{João Paulo Papa}
\institute{Advanced Institute for Artificial Intelligence (AI2)}

\begin{document}
% ####################################
% FIRST SLIDE 						:: \SliTit{This is the Title of the Talk}{A. B. Name}{Sprace}
% SUB-TITLE SLIDE 					:: \SliSubTit{<title>}{<explanation}
% SUB-SUB-TITLE SLIDE				:: \SliSubSubTit{<title>}{<explanation}
% SLIDE WITH TITLE 					:: \SliT{Title}{Content}
% SLIDE NO TITLE 						:: \Sli{Content} 
% SLIDE DOUBLE COLUMN WITH TITLE 	:: \SliDT{Title}{First Column}{Second Column}
% SLIDE DOUBLE COLUMN NO TITLE 		:: \SliD{First Column}{Second Column}
% SLIDE ADVANCED WITH TITLE 			:: \SliAdvT{Title}{Content}
% SLIDE ADVANCED NO TITLE 			:: \SliAdv{Content}
% SLIDE ADVANCED DOUBLE WITH TITLE 	:: \SliAdvDT{Title}{First Column}{Second Column}
% SLIDE ADVANCED DOUBLE NO TITLE 	:: \SliAdvD{First Column}{Second Column}
% SLIDE BLACK						:: \Black{ <Content> }
% SLIDE WHITE						:: \White{ <Content> }
% ITEMIZATION 						:: \begin{itemize}  \iOn{First} \iTw {Second} \iTh{Third} \end{itemize}
% COMMENT TEXT				 		:: \note{<comment>}
% SECTION 							:: \secx{Section} | \secxx{Sub-Section}
% BOLD SPRACE COLOR				:: \bfs{<text>}
% TABLE OF CONTENT					:: \tocitem{<title>}{<content>}
% LEFT ALIGN EQUATION				:: \begin{flalign*}  & <equation> &   \end{flalign*}
% CENTER ALIGN EQUATION	S			:: \begin{gather*} <equations>  \end{gather*}
% SLASH								:: \slashed{<>}
% BAR								:: \barr{<letter>} instead of \bar{<letter>}
% THEREFORE						:: use \portanto (larger and bold}
% 2 or 3 MATH SYMBOLS				:: \overset{<up>}{<down>} &  \underset{<below>}{\overset{<above>}{<middle>}}  
% INSERT TEXT IN FORMULA			:: \ins{<text>}
% EXERCISE							:: \exe{<exercise #>}{<exercise text>}
% SUGGESTED READING BOX			:: \sug{<references>}
% CITATION							:: \cittex{<citation>}
% CITATION DOUBLE COLUMN 			:: \cittexD{<citation>}
% TEXT POSITION						:: \texpos{<Xcm>}{<Ycm>}{<text>} origin = center of slide : x right | y down
% REFERENCE AT BOTTOM  S/D SLIDE		:: \refbotS{<reference>} \refbotD{<reference>}
% HIDDEN SLIDE						:: \hid
% COLOR BOX 						:: \blu{blue} + \red{rec} + \yel{yellow} + \gre{green} + \bege{beige}
% FRAME 							:: \fra{sprace} \frab{blue} \frar{red} + \fray{yellow} + \frag{green}		
% FIGURE 							:: \img{X}{Y}{<scale>}{Figure.png} 
% FIGURE							:: \includegraphics[scale=<scale>]{Figures/.png}
% FIGURE DOUBLE SLIDE NO TITLE		::  \img{-4}{0.5}{<scale>}{Figure.png} % Image 1st half
%									::  \img{4}{0.5}{<scale>}{Figure.png} % Image 2nd half
% FIGURE DOUBLE SLIDE WITH TITLE		::  \img{-4}{0}{<scale>}{Figure.png} % Image 1st half
%									::  \img{4}{0}{<scale>}{Figure.png} % Image 2nd half
% INCLUDING SWF (Flash)				:: \usepackage{media9} and \includemedia >> USE ACROBAT <<
%%%%%%%%%%%%%%%%%%%%%%%%%%%%%%%%%%%%%%%%%%%%%%%%%%
% ###############################################################################
% FIRST SLIDE
\SliTit{{\LARGE Modelo de Mistura de Gaussianas}}{Advanced Institute for Artificial Intelligence -- AI2}{https://advancedinstitute.ai}
%%%%%%%%%%%%%%%%%%%%%%%%%%%%%%%%%%%%%%%%%%%%%%%%%%
% ###############################################################################
% SLIDE SUB-TITLE
%\SliSubTit{Sub-Title}{Description}{}
%%%%%%%%%%%%%%%%%%%%%%%%%%%%%%%%%%%%%%%%%%%%%%%%%%
% ###############################################################################
%\SliSubSubTit{Sub-Sub-Title}{Description}
 %%%%%%%%%%%%%%%%%%%%%%%%%%%%%%%%%%%%%%%%%%%%%%%%%%


\SliT{Introdução}{

\justifying Modelo de Misturas de Gaussianas, do inglês \emph{Gaussian Misture Models} - GMMs, é uma técnica de \textbf{aprendizado não supervisionado} que pode ser entendida como uma generalização do $k$-médias. Ao invés de estimarmos os centroides de cada agrupamento, tentamos estimar também a forma e proporção de cada Gaussiana que compõe a mistura.\newline

\justifying \underline{Definição do problema:} seja ${\cal X}^1=\{\boldsymbol{x}_1,\boldsymbol{x}_2,\ldots,\boldsymbol{x}_m\}$ um conjunto de dados de treinamento tal que $\boldsymbol{x}_i\in\mathbb{R}^n$. Assumimos que as amostras são independentes e identicamente distribuídas (i.i.d.) a partir de uma função de densidade de probabilidade $p(\boldsymbol{x}_i)$, $\boldsymbol{x}_i\in{\cal X}^1$. Assumimos, também, que $p(\boldsymbol{x}_i)$ é uma \textbf{mistura finita} de $K$ componentes, ou seja:
\vspace{-0.1cm}
\begin{equation}
	p(\boldsymbol{x}_i|\boldsymbol{\theta}) = \sum_{k=1}^Kw_kp(\boldsymbol{x}_i|\boldsymbol{\theta}),
\end{equation}
em que $\theta_k = (w_k,\boldsymbol{\mu}_k,\boldsymbol{\Sigma}_k)$ corresponde aos parâmetros relativos à Gaussiana $k$ e $\sum_{k=1}^kw_k=1$.
}

\Sli{
As componentes são densidades Gaussianas multivariadas dadas por:

\begin{equation}
	p(\boldsymbol{x}_i|\theta_k) = \frac{1}{(2\pi)^{m^\prime_k/2}|\Sigma_k|^{1/2}}\exp\left\{-\frac{1}{2}(\boldsymbol{x}_i-\boldsymbol{\mu}_k)^T\boldsymbol{\Sigma}_k^{-1}(\boldsymbol{x}_i-\boldsymbol{\mu}_k)\right\}.
\end{equation}
No caso, os pesos $w_k$ representam a probabilidade que uma amostra $\boldsymbol{x}_i\in{\cal X}^1$ selecionada aleatoriamente tenha sigo gerada pela componente $k$. Vejamos o exemplo de uma mistura 1D com $k=3$.
}

\Sli{
\justifying Como não temos os rótulos, devemos utilizar alguma informação adicional para estimar $\boldsymbol{\theta} = \{\theta_1,\theta_2,\ldots,\theta_k\}$. Para tal tarefa, podemos empregar o conhecido algoritmo E-M (\emph{Expectation-Maximization}), o qual segue a mesma ideia da abordagem de máxima verossimilhança.\newline

\justifying A falta do rótulo dos dados caracteriza um \textbf{problema estatístico incompleto}, fazendo com que utilizemos as chamadas \textbf{variáveis latentes}, isto é, variáveis que \textbf{não são observadas}. Note que no caso do classificador Bayesiano temos o conjunto de rótulos, ou seja, observamos essas variáveis.
}

\Sli{
}

\end{document}