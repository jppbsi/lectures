%Fiquemos com Deus e Nossa Senhora!
%Sao Jose de Cupertino rogai por nos!!
% ### Uses XeLaTeX ### %
% ### Needs beamer-master ### %
\documentclass[aspectratio=169]{beamer} %. Aspect Ratio 16:9

\usetheme{AI2} % beamerthemeSprace.sty
\usepackage[portuguese]{babel}
\usepackage[utf8]{inputenc}
\usepackage[T1]{fontenc}
\usepackage{ragged2e,gensymb,bm,amsmath,amssymb}

\DeclareMathOperator*{\argmin}{arg\,min}
\DeclareMathOperator*{\argmax}{arg\,max}
\DeclareMathOperator{\sign}{sgn}

% DATA FOR FOOTER
\date{2021}
\title{- Kernel PCA}
\author{João Paulo Papa}
\institute{Advanced Institute for Artificial Intelligence (AI2)}

\begin{document}
% ####################################
% FIRST SLIDE 						:: \SliTit{This is the Title of the Talk}{A. B. Name}{Sprace}
% SUB-TITLE SLIDE 					:: \SliSubTit{<title>}{<explanation}
% SUB-SUB-TITLE SLIDE				:: \SliSubSubTit{<title>}{<explanation}
% SLIDE WITH TITLE 					:: \SliT{Title}{Content}
% SLIDE NO TITLE 						:: \Sli{Content} 
% SLIDE DOUBLE COLUMN WITH TITLE 	:: \SliDT{Title}{First Column}{Second Column}
% SLIDE DOUBLE COLUMN NO TITLE 		:: \SliD{First Column}{Second Column}
% SLIDE ADVANCED WITH TITLE 			:: \SliAdvT{Title}{Content}
% SLIDE ADVANCED NO TITLE 			:: \SliAdv{Content}
% SLIDE ADVANCED DOUBLE WITH TITLE 	:: \SliAdvDT{Title}{First Column}{Second Column}
% SLIDE ADVANCED DOUBLE NO TITLE 	:: \SliAdvD{First Column}{Second Column}
% SLIDE BLACK						:: \Black{ <Content> }
% SLIDE WHITE						:: \White{ <Content> }
% ITEMIZATION 						:: \begin{itemize}  \iOn{First} \iTw {Second} \iTh{Third} \end{itemize}
% COMMENT TEXT				 		:: \note{<comment>}
% SECTION 							:: \secx{Section} | \secxx{Sub-Section}
% BOLD SPRACE COLOR				:: \bfs{<text>}
% TABLE OF CONTENT					:: \tocitem{<title>}{<content>}
% LEFT ALIGN EQUATION				:: \begin{flalign*}  & <equation> &   \end{flalign*}
% CENTER ALIGN EQUATION	S			:: \begin{gather*} <equations>  \end{gather*}
% SLASH								:: \slashed{<>}
% BAR								:: \barr{<letter>} instead of \bar{<letter>}
% THEREFORE						:: use \portanto (larger and bold}
% 2 or 3 MATH SYMBOLS				:: \overset{<up>}{<down>} &  \underset{<below>}{\overset{<above>}{<middle>}}  
% INSERT TEXT IN FORMULA			:: \ins{<text>}
% EXERCISE							:: \exe{<exercise #>}{<exercise text>}
% SUGGESTED READING BOX			:: \sug{<references>}
% CITATION							:: \cittex{<citation>}
% CITATION DOUBLE COLUMN 			:: \cittexD{<citation>}
% TEXT POSITION						:: \texpos{<Xcm>}{<Ycm>}{<text>} origin = center of slide : x right | y down
% REFERENCE AT BOTTOM  S/D SLIDE		:: \refbotS{<reference>} \refbotD{<reference>}
% HIDDEN SLIDE						:: \hid
% COLOR BOX 						:: \blu{blue} + \red{rec} + \yel{yellow} + \gre{green} + \bege{beige}
% FRAME 							:: \fra{sprace} \frab{blue} \frar{red} + \fray{yellow} + \frag{green}		
% FIGURE 							:: \img{X}{Y}{<scale>}{Figure.png} 
% FIGURE							:: \includegraphics[scale=<scale>]{Figures/.png}
% FIGURE DOUBLE SLIDE NO TITLE		::  \img{-4}{0.5}{<scale>}{Figure.png} % Image 1st half
%									::  \img{4}{0.5}{<scale>}{Figure.png} % Image 2nd half
% FIGURE DOUBLE SLIDE WITH TITLE		::  \img{-4}{0}{<scale>}{Figure.png} % Image 1st half
%									::  \img{4}{0}{<scale>}{Figure.png} % Image 2nd half
% INCLUDING SWF (Flash)				:: \usepackage{media9} and \includemedia >> USE ACROBAT <<
%%%%%%%%%%%%%%%%%%%%%%%%%%%%%%%%%%%%%%%%%%%%%%%%%%
% ###############################################################################
% FIRST SLIDE
\SliTit{{\LARGE Kernel PCA}}{Advanced Institute for Artificial Intelligence -- AI2}{https://advancedinstitute.ai}
%%%%%%%%%%%%%%%%%%%%%%%%%%%%%%%%%%%%%%%%%%%%%%%%%%
% ###############################################################################
% SLIDE SUB-TITLE
%\SliSubTit{Sub-Title}{Description}{}
%%%%%%%%%%%%%%%%%%%%%%%%%%%%%%%%%%%%%%%%%%%%%%%%%%
% ###############################################################################
%\SliSubSubTit{Sub-Sub-Title}{Description}
 %%%%%%%%%%%%%%%%%%%%%%%%%%%%%%%%%%%%%%%%%%%%%%%%%%


\SliT{Introdução}{

\justifying \emph{Kernel PCA} (KPCA) é uma técnica que busca generalizar PCA visando realizar \textbf{redução não linear de dimensionalidade}. Lembrando que PCA assume que os dados estão em um espaço Euclidiano, ou seja, linear. Na hipótese que os dados estão em um espaço não linear, PCA não consegue garantir a sua otimalidade.\newline

\justifying Seja, então, um conjunto de dados ${\cal X}=\{\bm{x}_1,\bm{x}_2,\ldots,\bm{x}_m\}$ tal que $\bm{x}_i\in\mathbb{R}^n$ representa uma amostra no espaço de características original. A ideia do KPCA é fazer uso de uma função de mapeamento não linear, também conhecida por \emph{kernel}, $\phi(\bm{x})\in\mathbb{R}^{n^{\prime}}$, tal que $n^\prime>n$, visando mapear os dados para um espaço de maior dimensão. Geralmente, esses mapeamentos são custosos pois envolvem a projeção das amostras nestes espaços com maior dimensão. No entanto, podemos contornar isso com o \emph{kernel trick}, em que podemos \textbf{calcular o produto interno} no espaço de alta dimensão \textbf{sem a necessidade} de projeção dos dados.
}

\Sli{
\justifying Desta forma, KPCA realiza um aumento temporário da dimensionalidade do espaço para depois realizar a sua redução. \textbf{A ideia é incorporar não linearidades durante o processo de redução de dimensionalidade.}\newline

\justifying A primeira hipótese do KPCA é assumir que a média amostral após o mapeamento para o espaço de maior dimensão vale zero, ou seja:

\begin{equation}
	\bm{\mu}^\prime = \frac{1}{m}\sum_{i=1}^m\phi(\bm{x}_i) = 0,
\end{equation}
em que $\bm{\mu}^\prime\in\mathbb{R}^{n^{\prime}}$. Neste caso, KPCA assume que a média amostral está situada na origem do espaço de dados.
}

\Sli{
\justifying Já a matriz de covariância $\bm{\Sigma}\in\mathbb{R}^{n^\prime\times n^\prime}$ é dada por:

\begin{equation}
	\bm{\Sigma} = \frac{1}{m}\sum_{i=1}^m(\phi(\bm{x}_i)-\cancelto{0}{\bm{\mu}^\prime})(\phi(\bm{x}_i)-\cancelto{0}{\bm{\mu}^\prime})^T = \frac{1}{m}\sum_{i=1}^m\phi(\bm{x}_i)\phi(\bm{x}_i)^T.
\end{equation}
\justifying Temos que os autovetores $\bm{v}_k$ da matriz $\bm{\Sigma}$ são dados por:

$\forall k = 1,2,\ldots,n^\prime$.
}

\end{document}