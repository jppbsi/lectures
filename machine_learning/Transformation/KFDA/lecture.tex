%Fiquemos com Deus e Nossa Senhora!
%Sao Jose de Cupertino rogai por nos!!
%Honra teu Pai e tua Mãe!
% ### Uses XeLaTeX ### %
% ### Needs beamer-master ### %
\documentclass[aspectratio=169]{beamer} %. Aspect Ratio 16:9

\usetheme{AI2} % beamerthemeSprace.sty
\usepackage[portuguese]{babel}
\usepackage[utf8]{inputenc}
\usepackage[T1]{fontenc}
\usepackage{ragged2e,gensymb,bm,amsmath,amssymb}

\DeclareMathOperator*{\argmin}{arg\,min}
\DeclareMathOperator*{\argmax}{arg\,max}
\DeclareMathOperator{\sign}{sgn}

% DATA FOR FOOTER
\date{2021}
\title{- Kernel Fisher Discriminant Analysis}
\author{João Paulo Papa}
\institute{Advanced Institute for Artificial Intelligence (AI2)}

\begin{document}
% ####################################
% FIRST SLIDE 						:: \SliTit{This is the Title of the Talk}{A. B. Name}{Sprace}
% SUB-TITLE SLIDE 					:: \SliSubTit{<title>}{<explanation}
% SUB-SUB-TITLE SLIDE				:: \SliSubSubTit{<title>}{<explanation}
% SLIDE WITH TITLE 					:: \SliT{Title}{Content}
% SLIDE NO TITLE 						:: \Sli{Content} 
% SLIDE DOUBLE COLUMN WITH TITLE 	:: \SliDT{Title}{First Column}{Second Column}
% SLIDE DOUBLE COLUMN NO TITLE 		:: \SliD{First Column}{Second Column}
% SLIDE ADVANCED WITH TITLE 			:: \SliAdvT{Title}{Content}
% SLIDE ADVANCED NO TITLE 			:: \SliAdv{Content}
% SLIDE ADVANCED DOUBLE WITH TITLE 	:: \SliAdvDT{Title}{First Column}{Second Column}
% SLIDE ADVANCED DOUBLE NO TITLE 	:: \SliAdvD{First Column}{Second Column}
% SLIDE BLACK						:: \Black{ <Content> }
% SLIDE WHITE						:: \White{ <Content> }
% ITEMIZATION 						:: \begin{itemize}  \iOn{First} \iTw {Second} \iTh{Third} \end{itemize}
% COMMENT TEXT				 		:: \note{<comment>}
% SECTION 							:: \secx{Section} | \secxx{Sub-Section}
% BOLD SPRACE COLOR				:: \bfs{<text>}
% TABLE OF CONTENT					:: \tocitem{<title>}{<content>}
% LEFT ALIGN EQUATION				:: \begin{flalign*}  & <equation> &   \end{flalign*}
% CENTER ALIGN EQUATION	S			:: \begin{gather*} <equations>  \end{gather*}
% SLASH								:: \slashed{<>}
% BAR								:: \barr{<letter>} instead of \bar{<letter>}
% THEREFORE						:: use \portanto (larger and bold}
% 2 or 3 MATH SYMBOLS				:: \overset{<up>}{<down>} &  \underset{<below>}{\overset{<above>}{<middle>}}  
% INSERT TEXT IN FORMULA			:: \ins{<text>}
% EXERCISE							:: \exe{<exercise #>}{<exercise text>}
% SUGGESTED READING BOX			:: \sug{<references>}
% CITATION							:: \cittex{<citation>}
% CITATION DOUBLE COLUMN 			:: \cittexD{<citation>}
% TEXT POSITION						:: \texpos{<Xcm>}{<Ycm>}{<text>} origin = center of slide : x right | y down
% REFERENCE AT BOTTOM  S/D SLIDE		:: \refbotS{<reference>} \refbotD{<reference>}
% HIDDEN SLIDE						:: \hid
% COLOR BOX 						:: \blu{blue} + \red{rec} + \yel{yellow} + \gre{green} + \bege{beige}
% FRAME 							:: \fra{sprace} \frab{blue} \frar{red} + \fray{yellow} + \frag{green}		
% FIGURE 							:: \img{X}{Y}{<scale>}{Figure.png} 
% FIGURE							:: \includegraphics[scale=<scale>]{Figures/.png}
% FIGURE DOUBLE SLIDE NO TITLE		::  \img{-4}{0.5}{<scale>}{Figure.png} % Image 1st half
%									::  \img{4}{0.5}{<scale>}{Figure.png} % Image 2nd half
% FIGURE DOUBLE SLIDE WITH TITLE		::  \img{-4}{0}{<scale>}{Figure.png} % Image 1st half
%									::  \img{4}{0}{<scale>}{Figure.png} % Image 2nd half
% INCLUDING SWF (Flash)				:: \usepackage{media9} and \includemedia >> USE ACROBAT <<
%%%%%%%%%%%%%%%%%%%%%%%%%%%%%%%%%%%%%%%%%%%%%%%%%%
% ###############################################################################
% FIRST SLIDE
\SliTit{{\LARGE Kernel Fisher Discriminant Analysis}}{Advanced Institute for Artificial Intelligence -- AI2}{https://advancedinstitute.ai}
%%%%%%%%%%%%%%%%%%%%%%%%%%%%%%%%%%%%%%%%%%%%%%%%%%
% ###############################################################################
% SLIDE SUB-TITLE
%\SliSubTit{Sub-Title}{Description}{}
%%%%%%%%%%%%%%%%%%%%%%%%%%%%%%%%%%%%%%%%%%%%%%%%%%
% ###############################################################################
%\SliSubSubTit{Sub-Sub-Title}{Description}
 %%%%%%%%%%%%%%%%%%%%%%%%%%%%%%%%%%%%%%%%%%%%%%%%%%


\SliT{Introdução}{

\justifying A técnica de \emph{Kernel Fisher Discriminant Analysis} - (KFDA) é uma \textbf{generalização não linear} para LDA. Novamente, faremos uso dos \emph{kernels} para tornar LDA uma técnica de projeção não linear. Inicialmente, iremos abordar a versão para classificação com duas classes.\newline

\justifying Seja, então, ${\cal X} = \{(\bm{x}_1,y_1),(\bm{x}_2,y_2),\ldots,(\bm{x}_m,y_m)\}$ um conjunto de dados tal que $\bm{x}_i\in\mathbb{R}^n$ e $\phi:\mathbb{R}^n\rightarrow\mathbb{R}^{n^\prime}$ uma função de mapeamento não linear tal que $n^\prime>n$. Ademais, temos que ${\cal Y}=\{\omega_1,\omega_2\}$ de tal forma que $y_i\in\mathbb{\cal Y}$, $i=1,2,\ldots,m$. No KFDA busca-se maximizar o seguinte critério:

\begin{equation}
	J(\bm{w}) = \frac{\bm{w}^T\bm{S}^\phi_B\bm{w}}{\bm{w}^T\bm{S}^\phi_A\bm{w}}.
\end{equation}
Note que o critério é o mesmo que LDA. O que muda, porém, é a maneira com a qual temos que calcular as matrizes de espalhamento interclasses $\bm{S}^\phi_B$ e intraclasses $\bm{S}^\phi_A$.
}

\Sli{
\justifying Neste caso, temos que:

\begin{equation}
	\bm{S}^\phi_B = (\bm{\mu}_1^\phi-\bm{\mu}_2^\phi)(\bm{\mu}_1^\phi-\bm{\mu}_2^\phi)^T.
\end{equation}
Além disso, temos que:
\begin{equation}
	\bm{S}_A = \bm{S}_1+\bm{S}_2,
\end{equation}
em que
\begin{equation}
	\bm{S}_i = \sum_{\bm{x}_j\in\omega_1}(\phi(\bm{x}_j)-\bm{\mu}_i^\phi)(\phi(\bm{x}_j)-\bm{\mu}_i^\phi)^T,\ i\in\{1,2\}.
\end{equation}
Temos que a média de cada classe é dada por:
\begin{equation}
	\bm{\mu}_i^\phi = \frac{1}{m_i}\sum_{j=1}^{m_i}\phi(\bm{x}_j),\ i\in\{1,2\},
\end{equation}
em que $m_i$ denota a quantidade de elementos da classe $\omega_i$.
}

\end{document}