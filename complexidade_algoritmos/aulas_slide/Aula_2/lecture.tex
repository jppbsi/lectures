%Fiquemos com Deus e Nossa Senhora!
%Sao Jose de Cupertino rogai por nos!!
%Honra teu pai e tua mae!
% ### Uses XeLaTeX ### %
% ### Needs beamer-master ### %
\documentclass[aspectratio=169]{beamer} %. Aspect Ratio 16:9

\usetheme{AI2} % beamerthemeSprace.sty
\usepackage[portuguese]{babel}
\usepackage[utf8]{inputenc}
\usepackage[T1]{fontenc}
\usepackage{ragged2e,bm,oubraces,tikz,verbatim}
\DeclareMathOperator*{\argmin}{arg\,min}
\DeclareMathOperator*{\argmax}{arg\,max}

\newcommand*\mycirc[1]{%
  \begin{tikzpicture}
    \node[draw,circle,inner sep=2pt] {#1};
  \end{tikzpicture}}
  
\def\checkmark{\tikz\fill[scale=0.4](0,.35) -- (.25,0) -- (1,.7) -- (.25,.15) -- cycle;} 
  
% DATA FOR FOOTER
\date{2025}
\title{- Análise de Algoritmos}
\author{João Paulo Papa}
\institute{Análise de Algoritmos}

\begin{document}
% ####################################
% FIRST SLIDE 						:: \SliTit{This is the Title of the Talk}{A. B. Name}{Sprace}
% SUB-TITLE SLIDE 					:: \SliSubTit{<title>}{<explanation}
% SUB-SUB-TITLE SLIDE				:: \SliSubSubTit{<title>}{<explanation}
% SLIDE WITH TITLE 					:: \SliT{Title}{Content}
% SLIDE NO TITLE 						:: \Sli{Content} 
% SLIDE DOUBLE COLUMN WITH TITLE 	:: \SliDT{Title}{First Column}{Second Column}
% SLIDE DOUBLE COLUMN NO TITLE 		:: \SliD{First Column}{Second Column}
% SLIDE ADVANCED WITH TITLE 			:: \SliAdvT{Title}{Content}
% SLIDE ADVANCED NO TITLE 			:: \SliAdv{Content}
% SLIDE ADVANCED DOUBLE WITH TITLE 	:: \SliAdvDT{Title}{First Column}{Second Column}
% SLIDE ADVANCED DOUBLE NO TITLE 	:: \SliAdvD{First Column}{Second Column}
% SLIDE BLACK						:: \Black{ <Content> }
% SLIDE WHITE						:: \White{ <Content> }
% ITEMIZATION 						:: \begin{itemize}  \iOn{First} \iTw {Second} \iTh{Third} \end{itemize}
% COMMENT TEXT				 		:: \note{<comment>}
% SECTION 							:: \secx{Section} | \secxx{Sub-Section}
% BOLD SPRACE COLOR				:: \bfs{<text>}
% TABLE OF CONTENT					:: \tocitem{<title>}{<content>}
% LEFT ALIGN EQUATION				:: \begin{flalign*}  & <equation> &   \end{flalign*}
% CENTER ALIGN EQUATION	S			:: \begin{gather*} <equations>  \end{gather*}
% SLASH								:: \slashed{<>}
% BAR								:: \barr{<letter>} instead of \bar{<letter>}
% THEREFORE						:: use \portanto (larger and bold}
% 2 or 3 MATH SYMBOLS				:: \overset{<up>}{<down>} &  \underset{<below>}{\overset{<above>}{<middle>}}  
% INSERT TEXT IN FORMULA			:: \ins{<text>}
% EXERCISE							:: \exe{<exercise #>}{<exercise text>}
% SUGGESTED READING BOX			:: \sug{<references>}
% CITATION							:: \cittex{<citation>}
% CITATION DOUBLE COLUMN 			:: \cittexD{<citation>}
% TEXT POSITION						:: \texpos{<Xcm>}{<Ycm>}{<text>} origin = center of slide : x right | y down
% REFERENCE AT BOTTOM  S/D SLIDE		:: \refbotS{<reference>} \refbotD{<reference>}
% HIDDEN SLIDE						:: \hid
% COLOR BOX 						:: \blu{blue} + \red{rec} + \yel{yellow} + \gre{green} + \bege{beige}
% FRAME 							:: \fra{sprace} \frab{blue} \frar{red} + \fray{yellow} + \frag{green}		
% FIGURE 							:: \img{X}{Y}{<scale>}{Figure.png} 
% FIGURE							:: \includegraphics[scale=<scale>]{Figures/.png}
% FIGURE DOUBLE SLIDE NO TITLE		::  \img{-4}{0.5}{<scale>}{Figure.png} % Image 1st half
%									::  \img{4}{0.5}{<scale>}{Figure.png} % Image 2nd half
% FIGURE DOUBLE SLIDE WITH TITLE		::  \img{-4}{0}{<scale>}{Figure.png} % Image 1st half
%									::  \img{4}{0}{<scale>}{Figure.png} % Image 2nd half
% INCLUDING SWF (Flash)				:: \usepackage{media9} and \includemedia >> USE ACROBAT <<
%%%%%%%%%%%%%%%%%%%%%%%%%%%%%%%%%%%%%%%%%%%%%%%%%%
% ###############################################################################
% FIRST SLIDE
\SliTit{Aula 2 - Como Analisar Algoritmos}{Análise de Algoritmos}{}{João Paulo Papa (UNESP/Bauru)}
%%%%%%%%%%%%%%%%%%%%%%%%%%%%%%%%%%%%%%%%%%%%%%%%%%
% ###############################################################################
% SLIDE SUB-TITLE
%\SliSubTit{Sub-Title}{Description}{}
%%%%%%%%%%%%%%%%%%%%%%%%%%%%%%%%%%%%%%%%%%%%%%%%%%
% ###############################################################################
%\SliSubSubTit{Sub-Sub-Title}{Description}
 %%%%%%%%%%%%%%%%%%%%%%%%%%%%%%%%%%%%%%%%%%%%%%%%%%



\SliT{Introdução}{
\justifying Nesta aula iremos abordar os seguintes assuntos:

\begin{itemize}
	\item Características de um algoritmo.
	\item Como escrever e analisar um algoritmo.
\end{itemize}
}

\SliT{Características de um Algoritmo}{
\justify Um algoritmo deve possuir as seguintes características:

\begin{enumerate}
	\item Entrada: $0$ ou mais.
	\item Saída: pelo menos $1$ saída.
	\item Definição (\emph{definiteness}): ser predito com grande confiabilidade (sem ``passos mágicos").
	\item Finitude (\emph{Finiteness}): algoritmo deve \textbf{parar} em algum momento.
	\item Eficácia: prover uma \textbf{solução} que resolva o problema.
\end{enumerate}

}

\SliT{Como Escrever e Analisar um Algoritmo}{
\justify Tomemos como exemplo o algoritmo abaixo:\newline

\begin{minipage}{0.497\textwidth}
\emph{Algoritmo Swap(a,b)\{}
\begin{tabbing}
	\emph{\ \ \ temp = a;}\\
	\emph{\ \ \ a = b;}\\
	\emph{\ \ \ b = temp;}
\end{tabbing}
\}
\end{minipage}%%% to prevent a space
\begin{minipage}{0.37\textwidth}
O que levar em consideração?
\begin{enumerate}
	\item Tempo
	\item Espaço
	\item Consumo de banda
	\item Consumo de energia
	\item \# registradores CPU
\end{enumerate}
\null
\par\xdef\tpd{\the\prevdepth}
\end{minipage}
}

\Sli{
\justify Suponha que cada linha de nosso algoritmo tome uma (1) unidade de tempo (\textbf{não vamos considerar a chamada da função e os comandos \{ e \})}.\newline

\begin{minipage}{0.497\textwidth}
\emph{Algoritmo Swap(a,b)\{}
\begin{tabbing}
	\emph{\ \ \ temp = a;\ }\checkmark\\ 
	\emph{\ \ \ a = b;\ }\checkmark\\
	\emph{\ \ \ b = temp;\ }\checkmark
\end{tabbing}
\}

\end{minipage}%%% to prevent a space
\begin{minipage}{0.37\textwidth}
Podemos observar que nosso programa requer $3$ unidades de tempo, ou seja, $f(n)=3$. Temos que a complexidade de \textbf{tempo} nosso algoritmo é \textbf{constante}. 
\null
\par\xdef\tpd{\the\prevdepth}
\end{minipage}\newline

\justify Na prática, é o processo é um pouco mais \textbf{detalhado}. Por exemplo, suponha a seguinte sentença: $x=a*2+b*3$. Aqui temos $2$ multiplicações, $1$ adição e $1$ atribuição, ou seja, $4$ operações que iriam consumir $4$ unidades de tempo, ou seja, $f(n)=4$. No entanto, ao analisar a complexidade de algoritmos para $n\rightarrow\infty$, consideramos toda a sentença demandando apenas $1$ unidade de tempo.
}

\Sli{
\justify E com relação à complexidade de \textbf{espaço}? Dado que nosso algoritmo possui três variáveis, a complexidade de espaço é dada por $s(n)=3$. 

\justify Quanto temos uma complexidade \textbf{constante}, seja para tempo ou espaço, representamos ela como $\Theta(1)$. Isto significa que essa complexidade é muito pequena em relação à complexidade \textbf{geral} do algoritmo e, portanto, não tem influência sobre ela.
}

\Sli{
\justify Uma das técnicas mais utilizadas para calcular a complexidade de um algoritmo é o \textbf{Método de Contagem por Frequência}. Vejamos um exemplo abaixo:\newline

\begin{minipage}{0.437\textwidth}
\vspace{-1cm}
\emph{Algoritmo Soma(A,n)\{}
\begin{tabbing}
	\emph{\ \ \ s = 0;\ }\textcolor{red}{$\Theta(1)$}\\ 
	\emph{\ \ \ for(i=0; i < n; i++)\ }\textcolor{blue}{$\Theta(n+1)$}\\
	\emph{\ \ \ \ \ \ \ s = s+A[i];\ \textcolor{green}{$\Theta(n)$}}\\
	\emph{\ \ \ return s;\ }\textcolor{red}{$\Theta(1)$}
\end{tabbing}
\}

\end{minipage}%%% to prevent a space
\begin{minipage}{0.51\textwidth}
Em nosso algoritmo, o comando \emph{for} executa \textcolor{blue}{$\Theta(n+1)$} comparações e operação de soma é executada \textcolor{green}{$\Theta(n)$} vezes. Assim, a complexidade final é dada por:

\begin{align*}
	f(n) & = \textcolor{blue}{(n+1)} + \textcolor{green}{n} + \textcolor{red}{1} + \textcolor{red}{1}\\
	& = 2n+3\in \boxed{O(n)}.
\end{align*}
\null
\par\xdef\tpd{\the\prevdepth}
\end{minipage}
}

\Sli{
\justify E com relação à complexidade de espaço?\newline

\begin{minipage}{0.437\textwidth}
\vspace{1cm}
\emph{Algoritmo Soma(A,n)\{}
\begin{tabbing}
	\emph{\ \ \ s = 0;\ }\\ 
	\emph{\ \ \ for(i=0; i < n; i++)\ }\\
	\emph{\ \ \ \ \ \ \ s = s+A[i];\ }\\
	\emph{\ \ \ return s;\ }
\end{tabbing}
\}

\end{minipage}%%% to prevent a space
\begin{minipage}{0.51\textwidth}
Suponha que tenhamos um vetor $A$ com $n=55$ posições. Além disso, temos declaradas as variáveis $s$, $i$ e $n$. Desta forma, nossa complexidade de espaço é dada por $s(n)=n+1+1+1=n+3\in \boxed{O(n)}$.
\null
\par\xdef\tpd{\the\prevdepth}
\end{minipage}
}

\Sli{
\justify Vejamos um outro exemplo que calcula a soma de duas matrizes $A,B\in \mathbb{R}^{n\times n}$:\newline

\begin{minipage}{0.477\textwidth}
\vspace{-2.67cm}
\emph{Algoritmo Soma\_Matriz(A,B,n)\{}
\begin{tabbing}
	\emph{\ \ \ for(i=0; i < n; i++)\ }\textcolor{blue}{$\Theta(n+1)$}\\
	\emph{\ \ \ \ \ for(j=0; j < n; j++)\ }\textcolor{orange}{$\Theta(n+1)$}\\
	
	\emph{\ \ \ \ \ \ \ C[i] = A[i]+B[i];\ \textcolor{green}{$\Theta(n^2)$}}\\
	\emph{\ \ \ return C;\ }\textcolor{red}{$\Theta(1)$}
\end{tabbing}
\}

\end{minipage}%%% to prevent a space
\begin{minipage}{0.51\textwidth}
Em nosso algoritmo, o comando \emph{for} mais externo executa \textcolor{blue}{$\Theta(n+1)$} comparações. Para cada execução deste, o comando \emph{for} mais interno executa \textcolor{orange}{$\Theta(n+1)$} comparações. A operação de soma é executada \textcolor{green}{$\Theta(n^2)$} vezes. A complexidade final é dada por:

\begin{align*}
	f(n) & = \textcolor{blue}{(n+1)}\textcolor{orange}{(n+1)} + \textcolor{green}{n^2} + \textcolor{red}{1}\\
	& = n^2 + 2n + 1 + n^2+1\\
	&= 2n^2+2n+1\in \boxed{O(n^2)}.
\end{align*}
\null
\par\xdef\tpd{\the\prevdepth}
\end{minipage}
}

\Sli{
\justify E com relação à complexidade de espaço?\newline

\begin{minipage}{0.437\textwidth}
\emph{Algoritmo Soma\_Matriz(A,B,n)\{}
\begin{tabbing}
	\emph{\ \ \ for(i=0; i < n; i++)\ }\\
	\emph{\ \ \ \ \ for(j=0; j < n; j++)\ }\\
	
	\emph{\ \ \ \ \ \ \ C[i] = A[i]+B[i];}\\
	\emph{\ \ \ return C;\ }
\end{tabbing}
\}

\end{minipage}%%% to prevent a space
\begin{minipage}{0.51\textwidth}
\vspace{-1.47cm}
Temos três matrizes $A,B,C \in \mathbb{R}^{n\times n}$ e as variáveis $n$, $i$ e $j$. Desta forma, nossa complexidade de espaço é dada por $s(n)=3n^2+1+1+1=3n^2+3\in \boxed{O(n^2)}$.
\null
\par\xdef\tpd{\the\prevdepth}
\end{minipage}
}

\Sli{
\justify Vejamos um exemplo que calcula a multiplicação de duas matrizes $A,B\in \mathbb{R}^{n\times n}$:\newline

\begin{minipage}{0.477\textwidth}
\vspace{-1.9cm}
\emph{Algoritmo Multiplica\_Matriz(A,B,n)\{}
\begin{tabbing}
	\emph{\ \ \ for(i=0; i < n; i++)\ }\textcolor{blue}{$\Theta(n+1)$}\\
	\emph{\ \ \ \ \ for(j=0; j < n; j++)\ }\textcolor{orange}{$\Theta(n+1)$}\\
	\emph{\ \ \ \ \ \ \ C[i,j] = 0;\ \textcolor{green}{$\Theta(n^2)$}}\\
	\emph{\ \ \ \ \ \ \ for(k=0; k < n; k++)\ }\textcolor{purple}{$\Theta(n^2)\Theta(n+1)$}\\
	\emph{\ \ \ \ \ \ \ \ \ C[i,j] = C[i,j]+A[i,k]*B[k,j];\ \textcolor{green}{$\Theta(n^3)$}}\\
	\emph{\ \ \ return C;\ }\textcolor{red}{$\Theta(1)$}
\end{tabbing}
\}

\end{minipage}%%% to prevent a space
\begin{minipage}{0.51\textwidth}
Em nosso algoritmo, o comando \emph{for} mais externo executa \textcolor{blue}{$\Theta(n+1)$} comparações. Para cada execução deste, um segundo comando \emph{for} executa \textcolor{orange}{$\Theta(n+1)$} comparações. O comando de atribuição $C[i,j] = 0$ executa \textcolor{green}{$\Theta(n^2)$} vezes, e o  \emph{for} mais interno executa \textcolor{purple}{$\Theta(n^2)\Theta(n+1)$} vezes. Finalmente, o comando de multiplicação executa \textcolor{green}{$\Theta(n^3)$} vezes. A complexidade final é dada por:
\vspace{-.3cm}
\begin{align*}
	f(n) & = \textcolor{blue}{(n+1)}\textcolor{orange}{(n+1)} + \textcolor{green}{n^2} + \textcolor{purple}{n^2(n+1)} + \textcolor{green}{n^3} + \textcolor{red}{1}\\
	& = n^2 + 2n + 1 + n^2+2n^3+n^2+1\\
	&= 2n^3 + 3n^2+2n+1\in \boxed{O(n^3)}
\end{align*}
\null
\par\xdef\tpd{\the\prevdepth}
\end{minipage}
}

\Sli{
\justify Vejamos um outro exemplo:\newline

\begin{minipage}{0.437\textwidth}
\emph{Algoritmo Soma\_Matriz\_2(A,B,n)\{}
\begin{tabbing}
	\emph{\ \ \ for(i=0; i < n; i++)\ }\\
	\emph{\ \ \ \ \ for(j=0; j < i; j++)\ }\\
	
	\emph{\ \ \ \ \ \ \ C[i] = A[i]+B[i];}\\
	\emph{\ \ \ return C;\ }
\end{tabbing}
\}\\\ \\
Assim, temos que:
\vspace{-.1cm}
\begin{align*}
	f(n)&=1+2+\ldots+n\\
	&=\frac{n(n+1)}{2}\in \boxed{O(n^2)}
\end{align*}
\end{minipage}%%% to prevent a space
\begin{minipage}{0.51\textwidth}
\vspace{-1.47cm}
Este é um exemplo um pouco mais complexo que requer uma análise mais \textbf{cuidadosa}. Vamos analisar alguns valores de $n$ e verificar quantas vezes o \emph{for} mais interno executa:

\begin{table}
\begin{tabular}{c|c}
$n$ & \# vezes\\\hline
0 & 0\\
1 & 1\\
2 & 2\\
$\vdots$ & $\vdots$\\
$n$ & $n$\\
\end{tabular}
\end{table}

\null
\par\xdef\tpd{\the\prevdepth}
\end{minipage}
}

\Sli{
\justify Vejamos um outro exemplo:\newline

\begin{minipage}{0.437\textwidth}
\emph{Algoritmo Soma\_p(A,B,n)\{}
\begin{tabbing}
	\emph{\ \ \ p=0\ }\\
	\emph{\ \ \ \ \ for(i=1; p $\leq$ n; i++)\ }\\
	\emph{\ \ \ \ \ \ \ p = p + i;}\\
	\emph{\ \ \ return p;\ }
\end{tabbing}
\}\\\ \\
\end{minipage}%%% to prevent a space
\begin{minipage}{0.51\textwidth}
\vspace{-1.01cm}
Analisando os valores de $n$ e $p$ temos que:

\begin{table}
\begin{tabular}{c|c}
$n$ & $p$\\\hline
1 & 0+1\\
2 & 0+1+2\\
3 & 0+1+2+3\\
$\vdots$ & $\vdots$\\
$k$ & $0+1+2+\ldots+k$\\
\end{tabular}
\end{table}

\null
\par\xdef\tpd{\the\prevdepth}
\end{minipage}
}

\Sli{
\justify A única certeza que temos é que o algoritmo vai parar quando $p>n$.

\justify Temos que $p=0+1+2+\ldots+k=\frac{k(k+1)}{2}>n$. Como $\frac{k(k+1)}{2}\in O(k^2)$, temos que:

\begin{equation*}
k^2 > n \implies k > \sqrt{n} \in \boxed{O(\sqrt{n})}.	
\end{equation*}
}

\Sli{
\justify Vejamos um outro exemplo:\newline

\begin{minipage}{0.437\textwidth}
\emph{Algoritmo Mul\_p(A,B,n)\{}
\begin{tabbing}
	\emph{\ \ \ p=1\ }\\
	\emph{\ \ \ for(i=1; i < n; i = i*2)\ }\\
	\emph{\ \ \ \ \ \ \ p = p * i;}\\
	\emph{\ \ \ return p;\ }
\end{tabbing}
\}\\\
\justify O laço vai parar quando $i\geq n$. No entanto, $i=n$ é suficiente para o laço parar. Assim, temos que $i=n=2^k\implies k=\log_2n\in \boxed{O(\log_2n)}$.
\end{minipage}%%% to prevent a space
\begin{minipage}{0.51\textwidth}
\vspace{-0.1cm}
Analisando os valores de $i$ temos que:

\begin{table}
\begin{tabular}{c|c}
$n$ & \# vezes\\\hline
$2^0=1$ & 0\\
$2^1=2$ & 1\\
$2^2=4$ & 2\\
$2^3=8$ & 3\\
$\vdots$ & $\vdots$\\
$2^k$ & $k$\\
\end{tabular}
\end{table}

\null
\par\xdef\tpd{\the\prevdepth}
\end{minipage}
}

\Sli{
\justify Vejamos um outro exemplo:\newline

\begin{minipage}{0.437\textwidth}
\emph{Algoritmo Mul\_p\_2(A,B,n)\{}
\begin{tabbing}
	\emph{\ \ \ p=1\ }\\
	\emph{\ \ \ for(i=n; i $\geq$ 1; i = i/2)\ }\\
	\emph{\ \ \ \ \ \ \ p = p * i;}\\
	\emph{\ \ \ return p;\ }
\end{tabbing}
\}\\\
\justify O laço vai parar quando $i<n$, ou seja, quando $i<2^{k-1}$. Assumindo que $i=n=2^{k-1}\implies n=\frac{2^k}{2}\implies2n=2^k\implies k=\log_2 2n\in \boxed{O(\log_2n)}$.
\end{minipage}%%% to prevent a space
\begin{minipage}{0.51\textwidth}
\vspace{-0.3cm}
Analisando os valores de $i$ temos que:

\begin{table}
\begin{tabular}{c|c}
$n$ & \# vezes\\\hline
$2^{1-1}\impliedby$1 & 1\\
$2^{2-1}\impliedby$2 & 2\\
$2^{4-1}\impliedby$8 & 4\\
$2^{5-1}\impliedby$16 & 5\\
$\vdots$ & $\vdots$\\
$2^{k-1}$ & $k$\\
\end{tabular}
\end{table}

\null
\par\xdef\tpd{\the\prevdepth}
\end{minipage}
}

\Sli{
\justify Vejamos um outro exemplo:\newline

\begin{minipage}{0.437\textwidth}
\emph{Algoritmo Exemplo(A,B,n)\{}
\begin{tabbing}
	\emph{\ \ \ p=0\ }\\
	\emph{\ \ \ for(i=1; i < n; i = i*2)\ }\\
	\emph{\ \ \ \ \ \ \ p = p + 1; }\textcolor{red}{$\Theta(\log n)$}\\
	\emph{\ \ \ for(j=1; j < p; j = j*2)\ }\\
	\emph{\ \ \ \ \ \ \ x = x + 1; }\textcolor{red}{$\Theta(\log p)$}
\end{tabbing}
\}\\\
\end{minipage}%%% to prevent a space
\begin{minipage}{0.51\textwidth}
\vspace{-0.3cm}
Neste caso, temos que a complexidade do primeiro laço é dada por $\Theta(\log n))$ e a do segundo por $\Theta(\log p)$. No entanto, temos que $p=\log n$ (\textbf{valor de $p$ é igual ao número de vezes que o primeiro laço executa)}. Assim, a complexidade final é dada por:

\begin{align*}
	f(n) &= \log n + \log p\\
	&=\log n + \log (\log n)\in \boxed{O(\log n)}
\end{align*}
\null
\par\xdef\tpd{\the\prevdepth}
\end{minipage}
}

\Sli{
\justify Podemos, então derivar complexidades para laços de acordo com o incremento do contador:

\begin{itemize}
	\item \emph{\ \ \ for(i=0; i < n; i = i++)\ }: \textcolor{red}{$O(n)$}
	\item \emph{\ \ \ for(i=0; i < n; i = i+2)\ }: \textcolor{red}{$\frac{n}{2}O(n)\in O(n)$}
	\item \emph{\ \ \ for(i=n; i > 1; i--)\ }\textcolor{red}{$O(n)$}
	\item \emph{\ \ \ for(i=1; i < n; i = i*2)\ }\textcolor{red}{$O(\log_2n)$}
	\item \emph{\ \ \ for(i=1; i < n; i = i*3)\ }\textcolor{red}{$O(\log_3n)$}
	\item \emph{\ \ \ for(i=n; i > 1; i = i/2)\ }\textcolor{red}{$O(\log_2n)$}
	
\end{itemize}
}


\end{document}